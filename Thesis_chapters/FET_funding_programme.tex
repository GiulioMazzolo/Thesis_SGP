\chapter{FET in Horizon 2020} \label{FET_in_Horizon_2020}
As discussed in chapter \ref{The_role_of_science_communication}, science communication plays a major role in the development of democratic societies. This is particularly true in the case of publicly funded research. Informing non-scientists of scientific investigations supported with public funding fulfils not only the need for the acquisition of the scientific citizenship, but also the citizens' right to know about the use of tax money. For these reasons, EU-funded research projects invest part of their budget in communication activities. 

This thesis presents an analysis of the online communication strategies followed by research projects financed by the European Commission within the Future and Emerging Technologies (FET) initiative, a branch of the Horizon 2020 funding programme. In particular, it compares the communication efforts of two FET research lines: the design of high-performing computers and the development of quantum technologies. The analysis and the results are presented in the second part of the thesis, whereas a description of the FET programme and of the aforementioned research lines is given in this chapter. 

The chapter is structured as follows. Section \ref{The_Horizon_2020_programme} illustrates the Horizon 2020 initiative. Section \ref{The_FET_programme} provides a description of FET in the framework of Horizon 2020. Sections \ref{FET_and_high-performing_computing} and \ref{FET_and_quantum_technologies} summarise the FET effort towards the development of high-performing computers and quantum technologies.

\section{The Horizon 2020 programme} \label{The_Horizon_2020_programme}
Horizon 2020 is the biggest research and innovation programme funded by the European Union to date. It targets a  sustainable societal and economic growth via the development and application of scientific research. The available budget totals nearly \euro 80 billion over a seven-year period (from 2014 to 2020) \cite{Horizon2020}.

Horizon 2020 is Europe's eighth research and innovation programme in chronological order \cite{FP4,FP5,FP6,FP7}. The first one was launched in 1984. Duration and allocated budget of each research and innovation programme are shown in figure \ref{FP_funds}.

\begin{figure}[!t] 
 \begin{center}
 \includegraphics[scale=0.5]{Images/FP_funds.png}
 \caption{Duration and allocated budget of the European research and innovation programmes (also known as Framework Programmes, FP). Data from \cite{FPBudget}.}
 \label{FP_funds}
 \end{center}
\end{figure}

\begin{figure}[!t] 
 \begin{center}
 \includegraphics[scale=0.43]{Images/H2020_budget_breakdown.png}
 \caption{Budget breakdown of the Horizon 2020 programme. Original image in \cite{H2020Budget}.}
 \label{H2020_budget_breakdown}
 \end{center}
\end{figure}

Any natural or legal persons (e.g. universities, research organisations and companies) can apply for Horizon 2020 funding. Applications must fit into one of the following categories: 

\begin{itemize}
 \item \textbf{Excellent Science:} this initiative supports the excellence of European scientific research on a global level and in a variety of fields \cite{ExcellentScience}.
 \item \textbf{Industrial Leadership:} this class of projects targets the development of technological innovations for the future market and the growth of European small and medium enterprises \cite{IndustrialLeadership}.
 \item \textbf{Societal Challenges:} this category focuses on priorities of the European society such as health, education, food and energy supply by combining knowledge and methods from disparate scientific fields\cite{SocietalChallenges}.  
 \item \textbf{European Institute for Innovation and Technology:} this institute is an independent European body promoting synergies in the fields of education, research and business \cite{EIT}. 
 \item \textbf{Euratom:} this pillar funds nuclear research in the framework of the decarbonisation of the energy supply \cite{Euratom}.
\end{itemize}
The Horizon 2020 budget breakdown into the aforementioned lines of action is shown in figure \ref{H2020_budget_breakdown}.

\section{The FET programme} \label{The_FET_programme}
As mentioned in section \ref{The_Horizon_2020_programme}, one of the Horizon 2020 initiatives is the Excellent Science programme. This line of action supports researchers and institutions developing new science and cutting-edge technology. The goal is to keep European research at the forefront of scientific innovation and discover applications to improve the citizens' life and ensure economical growth.  

\begin{figure}[!t] 
 \begin{center}
 \includegraphics[scale=0.43]{Images/Country_participation_in_H2020_FET_projects.jpg}
 \caption{Participants in the Horizon 2020 FET programme on a country basis as of June 2016. The numbers correspond to FET funding in million Euro. The colours indicate the number of participants. Adapted from image in \cite{FETParticipation}.}
 \label{Country_participation_in_H2020_FET_projects}
 \end{center}
\end{figure}

Excellent Science is based on the following pillars: 

\begin{itemize}
 \item \textbf{European Research Council:} it distributes funding in every research field to single scientists and with the requirement of scientific excellence \cite{ERC}.  
 \item \textbf{Future and Emerging Technologies (FET):} it finances collaborative research exploring visionary and radically new investigation lines \cite{FET}. 
 \item \textbf{Marie Sk\l{}odowska-Curie Actions:} this initiative assigns grants to researchers at any stage of their career and encourages mobility between countries and fields of expertise \cite{MSCA}. 
 \item \textbf{Research infrastructure:} it promotes the creation of transnational networks of research infrastructures as well as the training of qualified staff \cite{ResearchInfrastructure}. 
\end{itemize}
The estimated final budget breakdown of Excellent Science is reported in table \ref{FET_budget_breakdown}.

\begin{table}[t]
 \begin{center}
  \begin{tabular}{cc}
   \hline 
   \hline
   Line of action & Estimated final budget \\ 
   \hline
   \hline
   ERC & 13.1 \\
   FET & 2.7 \\
   MSCA & 6.2 \\
   RI & 2.5 \\
   \hline
   \hline
  \end{tabular}
 \end{center} 
 \caption{Estimated final budget breakdown of the Excellent Science initiative. ERC stands for European Research Council; FET for Future and Emerging Technologies; MSCA for Marie Sk\l{}odowska-Curie Actions; RI for Research infrastructure. Budgets are in billion Euro. Data from \cite{H2020Budget}.}
\label{FET_budget_breakdown} 
\end{table}

This thesis focuses on the online communication activity of the FET projects funded within Horizon 2020 by the time of writing. The list of these projects is available in appendix \ref{List_of_FET_projects}. The distribution of projects participants per country as of June 2016 is shown in figure \ref{Country_participation_in_H2020_FET_projects}.

The FET programme comprises three calls for applications: FET Open, FET Proactive and FET Flagship \cite{FETOpen,FETProactive,FETFlagship}.

\subsubsection{FET Open}
The FET Open call is not bound to one specific investigation theme. However, submitted research proposals must satisfy the following ``gatekeepers": scientific and technological breakthrough; foundational; novelty; high-risk; long-term vision; interdisciplinary. 

FET Open promotes the Coordination and Support Actions (CSA) as well. These aim at identifying and fulfilling the optimal conditions for FET-related collaborative investigation. One CSA type of action is the FET Innovation Launchpad. This explores possible economical and societal applications of FET results \cite{FETLaunchpad}. The list of Horizon 2020 projects funded within the FET Innovation Launchpad action is reported in appendix \ref{Specific_lists_of_FET_projects}. 

\subsubsection{FET Proactive}
The FET Proactive call nurtures synergies on specific research lines by bringing together scientists from interdisciplinary fields. The considered research lines are not ready for the market yet.    

Currently, FET Proactive comprises three calls related to ``Boosting emerging technologies" and three under ``High Performance Computing". Given its relevance for this thesis, the ``High Performance Computing" FET Proactive call is illustrated in section \ref{FET_and_high-performing_computing}. 

FET Proactive invests resources also in identifying investigation roadmaps, designing and distributing material for educational purposes and disseminating FET results among interested stakeholders.  

\subsubsection{FET Flagship}
FET Flagships are Europe's main research effort. They are large-scale, decade-long projects with budgets totalling one billion Euro each. The ultimate goal is to shed light on key scientific themes and apply the results to European society. To date, three FET Flagships have been approved in the Horizon 2020 programme: 

\begin{itemize}
 \item \textbf{Human Brain Project}, targeting groundbreaking steps forward in neuroscience \cite{HBP}.
 \item \textbf{Graphene}, exploring graphene's properties and possible applications \cite{Graphene}.
 \item \textbf{Quantum Technologies}, aiming to develop innovative technologies based on the laws of quantum physics.
\end{itemize}
The Human Brain Project and Graphene Flagships started in April 2016. The Quantum Technologies Flagship will start in 2018. Given the relevance of quantum technologies for this thesis, a concise description of the motivations and objectives behind their development is available in section \ref{FET_and_quantum_technologies}.

\section{FET and high-performing computing} \label{FET_and_high-performing_computing}
Current and future scientific and engineering challenges require increasing levels of computational performances. The demand can be satisfied via the construction of large computer clusters and the development of suitable programming languages. The former provide higher computational power for parallel calculations, the latter an optimal exploitation of the clusters' resources. The use of such practices is known as high-performing computing (HPC) \cite{Hager}.

In terms of increasing computational power, one major HPC goal is the transition from the peta- to the exascale. This corresponds to the increase from $10^{15}$ floating point operations per second, i.e. the limit of present-day most powerful supercomputers, to $10^{18}$. The upgrade to the exascale is motivated by its major impact on all scientific fields over the next decades \cite{Vetter}. 

As mentioned in section \ref{The_FET_programme}, the FET HPC research line is funded within the ``High Performance Computing" Proactive call \cite{HPC}. This call comprises three initiatives: \textit{i}) co-design of HPC systems and applications; \textit{ii}) transition to exascale computing; and \textit{iii}) exascale HPC ecosystem development. The main goals of the three initiatives are to develop the next-generation exascale high-performing computers and to provide access to the resources offered by these supercomputers. The list of Horizon 2020 FET projects active in HPC is available in appendix \ref{Specific_lists_of_FET_projects}.

\begin{figure}[!t] 
 \begin{center}
 \includegraphics[scale=0.23]{Images/EuroHPC.jpg}
 \caption{European countries signatories of the EuroHPC declaration as of October 2017. Adapted from image in \cite{EuroHPC_countries}.}
 \label{EuroHPC_image}
 \end{center}
\end{figure}

The FET ``High Performance Computing" call is only part of the overarching European effort for the development of HPC. One major initiative in this direction is EuroHPC, a transnational framework aiming to construct two supercomputers based on European technology by the beginning of the next decade \cite{EuroHPC}. The EuroHPC infrastructure and its computational resources will be available to support disparate kinds of communities, such as researchers, industry and the public sector. The creation of the European HPC ecosystem is endorsed by the countries in figure \ref{EuroHPC_image}, signatories of the EuroHPC declaration \cite{EuroHPC_declaration}.  

\section{FET and quantum technologies} \label{FET_and_quantum_technologies}
Quantum technologies arise from applications of quantum physics. They are an important research topic on a global level for their potential to revolutionise human society.

The so-called first quantum revolution started at the beginning of the past century with the development of the quantum theory. The growing understanding of the atomic world led to the birth of new disciplines, such as informatics and microelectronics, and to the construction of countless fundamental tools and electronic devices. Examples range from computers and cameras to lasers and photocopy machines. The first quantum revolution played a key role in starting the knowledge era of human society.

It is believed that the second quantum revolution will be driven by the ability acquired by humankind to actively engineer the quantum world to its own purposes \cite{Dowling}. This is expected to lead to a complete new class of technologies which would reshape our society. One example is the development of quantum computers. If successfully developed, such machines will be far more powerful than any present and future computer based on classical architectures \cite{Rieffel}. The urge for Europe to stay at the forefront of the second quantum revolution is outlined in the so-called Quantum Manifesto \cite{QuantumManifesto}.

The development of quantum technologies is a central objective of the FET programme.  The list of Horizon 2020 FET projects in this field is reported in appendix \ref{Specific_lists_of_FET_projects}. Their activity is supported by the ERANET Cofund in Quantum Technologies, a FET Proactive initiative fostering synergies and partnerships among researchers and other stakeholders \cite{ERANET}. Finally, as mentioned in section \ref{The_FET_programme}, one dedicated flagship initiative will be launched in 2018. 

\section{Chapter summary} 
In this chapter, the following items have been discussed:

\begin{enumerate}
 \item Horizon 2020 is the largest research funding programme of the European Union. It is planned to run from 2014 to 2020 and has a total budget of nearly \euro 80 billion. 
 \item One branch of Horizon 2020 is the Future and Emerging Technologies (FET) programme. The FET call finances visionary research projects targeting scientific breakthroughs and the development and application of radically new technologies. The estimated FET final budget will total nearly \euro 3 billion. 
 \item One central goal of the FET initiative is the development of high-performing computers. This investigation line targets a power increase in modern supercomputers of three orders of magnitude (from $10^{15}$ to $10^{18}$ floating point operations per second). The upgrade from the peta- to the exascale will provide unprecedented computational resources in practically all scientific fields.
 \item Another major effort of the FET programme is the development of quantum technologies. In particular, a FET Flagship on quantum technologies has been approved in 2016 by the European Commission and will start in 2018. Allocated funds sum up to \euro 1 billion.    
\end{enumerate}