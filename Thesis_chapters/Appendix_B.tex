\chapter{Specific lists of FET projects} \label{Specific_lists_of_FET_projects}
Disparate groups consisting of projects in appendix \ref{List_of_FET_projects} were considered in this thesis. The motivations for the identification of the groups are outlined in section \ref{Data_set}. The projects in each of the groups are listed below.

\section{Disregarded projects} \label{Disregarded_projects}
The following groups of projects were not considered for the analyses in this thesis.

\begin{description}
 \item [Flagship projects:] GrapheneCore1 and HBP SGA1.
 \item [Launchpad projects:] aPad, CASEK, CF-Web, D-Noise, DMS, ENTIMENT, I2C8, INTERLACE, PhySense, Qdet, QUSMI, ROMA, SensAgain, SmartNurse, WASPSNEST and WhiteRabbit.
 \item [Started after 1 February 2017:] CATCH-U-DNA, EuroEXA, FEMTOTERABYTE, Qdet and QUSMI. The DEEP-EST project was also launched after 1 February 2017. Nevertheless, it was considered for the analysis as it could make use of the channels activated for the DEEP and DEEP-ER projects, see section \ref{Data_set}. Note that Qdet and QUSMI are also Launchpad projects. 
\end{description}

\section{Investigated classes}
This thesis presents a comparison of the use of online communication channels made by projects active in high-performance computing and in the development of quantum technologies. The projects in the two classes are listed below. 

\subsubsection{High performance computing}
ALLScale, ANTAREX, ComPat, DEEP-EST, ECOSCALE, ESCAPE, EuroLab-4-HPC, ExaFLOW, ExaHyPE, ExaNeSt, ExaNoDe, ExCAPE, EXDCI, EXTRA, greenFLASH, INTERTWINE, MANGO, Mont-Blanc 3, NEXTGenIO, NLAFET, READEX and SAGE. The  EuroEXA project was not considered as it was launched after 1 February 2017, see section \ref{Disregarded_projects}.

\subsubsection{Quantum technologies}
AQuS, MaQSens, NanOQTech, QCUMbER, QuantERA, QUCHIP, QUIC, QuProCS, RYSQ and ULTRAQCL. The Qdet and QUSMI projects were not considered as they were launched after 1 February 2017 and are two Launchpad projects, see section \ref{Disregarded_projects}.