\chapter{Specific lists of FET projects} \label{Specific_lists_of_FET_projects}
Disparate groups of projects in Appendix \ref{List_of_FET_projects} were considered in this thesis. The motivations for the identification of the groups are outlined in section \ref{Data_set}. The projects in each of the groups are listed below.

\section{Disregarded projects}
The following groups of projects were not considered for the analyses preseted in this thesis:

\subsubsection{Flagship projects}
GrapheneCore1 and HBP SGA1.

\subsubsection{Launchpad projects}
aPad, CASEK, CF-Web, D-Noise, DMS, ENTIMENT, I2C8, INTERLACE, PhySense, Qdet, QUSMI, ROMA, SensAgain, SmartNurse, WASPSNEST and WhiteRabbit.

\subsubsection{Started after 1st February 2017}
CATCH-U-DNA, EuroEXA and FEMTOTERABYTE. The DEEP-EST project was also launched after 1st February 2017. Nevertheless, it was considered for the analysis as it had already activated several communication channels by the time of writing.

\section{Investigated classes}
This thesis presents a comparison of the use of online communication channels made by projects active in high-performing computing and in the development of quantum technologies. The projects in the two classes are listed below. 

\subsubsection{High performing computing (HPC)}
ALLScale, ANTAREX, ComPat, DEEP-EST, ECOSCALE, ESCAPE, EuroLab-4-HPC, ExaFLOW, ExaHyPE, ExaNeSt, ExaNoDe, ExCAPE, EXDCI, EXTRA, greenFLASH, INTERTWINE, MANGO, Mont-Blanc 3, NEXTGenIO, NLAFET, READEX and SAGE. The  EuroEXA project was not considered as it was launched after 1st February 2017, see above.

\subsubsection{Quantum technologies (QT)}
AQuS, MaQSens, NanOQTech, QCUMbER, QuantERA, QUCHIP, QUIC, QuProCS, RYSQ and ULTRAQCL. The QUSMI e Qdet projects were not considered as they were launched after 1st February 2017, see above.