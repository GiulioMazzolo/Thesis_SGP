\chapter*{Synopsis}
This thesis is divided into two parts. Part I (chapters 1 and 2) presents and overview of the importance of science communication in modern society and a description of the Future and Emerging Technologies (FET) science funding programme of the European Union. This part is not based on original of the author. Part II (chapters 3 - 5) presents the analyses performed by the author and the results. A summary of each chapter is provided below.

\section*{Part I: Introduction}

\begin{description}
 \item [Chapter 1: The role of science communication] This chapter highlights the societal importance of science communication. The main items discussed in the chapter are: \emph{i}) the knowledge era, \emph{ii}) societal and economical challenges of the knowledge era, \emph{iii}) the scientific citizenship as a solution to the challenges of the knowledge era, and \emph{iv}) science communication and its role for the creation of a democratic information society.

 \item [Chapter 2: FET in Horizon 2020] This chapter describes the FET funding programme and two of its research lines: the development of high performing computing (HPC) and quantum technologies (QT). The main items discussed in the chapter are: \emph{i}) the Horizon 2020 funding programme, \emph{ii}) the FET programme as one branch of Horizon 2020, and \emph{iii}) the HPC and QT efforts of the FET funding initiative.
\end{description}

\section*{Part II: Analysis and results}

\begin{description}
 \item[Chapter 3: FET projects and social media] This chapter analyses the presence of FET projects on online social media. The main items discussed in the chapter are: \emph{i}) the fraction of FET projects active on the considered social platforms, \emph{ii}) the comparison of the presence of HPC and QT projects on social networks, and \emph{iii}) the impact of the budget available to projects on the number of online channels activated for science communication campaigns.

 \item[Chapter 4: HPC projects on Twitter] This chapter investigates the activity of HPC research projects on Twitter. The main items discussed in the chapter are: \emph{i}) the analysis of the statistics collected from the Twitter profiles of the FET HPC initiatives, \emph{ii}) the identification of the most influential projects, and \emph{iii}) the virality of the Twitter conversations mentioning FET HPC projects.

 \item[Chapter 5: The Twitter potential reach of QT projects] This chapter presents an estimate of the community reachable by QT projects if they were active on Twitter. The main items discussed in the chapter are: \emph{i}) the spread of hashtags representative of conversations on HPC and QT projects, and \emph{ii}) the comparison of the aforementioned statistics to assess the Twitter audience interested in QTs.
\end{description}

\noindent
This work was done under the supervision of Dr. Daniela Ovadia. Questions and comments to be sent to giulio.mazzolo@gmail.com.