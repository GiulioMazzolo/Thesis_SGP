\chapter*{Synopsis}
This thesis is divided into two parts. Part I (Chapters 1 and 2) reports introductive material on the importance of science communication in modern society and on the Future and Emerging Technologies (FET) science funding programme of the European Union. This part does not include original contribution from the author. Part II (Chapters 3 - 5) presents the analyses performed and it is based on original work of the author. A summary of each chapter is presented below.

\section*{Part I: Introduction}

\begin{description}
 \item [Chapter 1: The role of science communication] This chapter presents an overview of the societal importance of science communication. The main items discussed in the chapter are: \emph{i}) the knowledge era and its characteristics, \emph{ii}) societal and economical challenges arisen in the knowledge era, \emph{iii}) the concept of scientific citizenship and how this may help tackle the problems of the knowledge era, and \emph{iv}) science communication as a tool toward the creation of a democratic information society.

 \item [Chapter 2: FET in Horizon 2020] This chapter focuses on the FET funding programme and on two research lines financed within this initiative: high performing computing (HPC) and quantum technologies (QT). The main items discussed in the chapter are: \emph{i}) the Horizon 2020 funding programme, \emph{ii}) the FET programme as one branch of Horizon 2020, and \emph{iii}) the HPC and QT efforts in the FET funding initiative.
\end{description}

\section*{Part II: Analysis and results}

\begin{description}
 \item[Chapter 3: FET projects and social media] This chapter describes the presence of FET projects on online social media. The main items discussed in the chapter are: \emph{i}) fraction of FET projects active on disparate social platforms, \emph{ii}) comparison of the social presence of HPC and QT projects, and \emph{iii}) the impact of the budget available to projects on the number of social channels considered for science communication campaigns.

 \item[Chapter 4: HPC projects on Twitter] This chapter investigates the activity of HPC research projects on Twitter. The main items discussed in the chapter are: \emph{i}) an analysis of the statistics collected from the Twitter profiles of the FET HPC initiatives, \emph{ii}) the most influential projects, \emph{iii}) the virality of the Twitter conversations mentioning FET HPC projects.

 \item[Chapter 5: The Twitter potential reach of QT projects] This chapter presents an estimate of the community which could be reached by QT projects if they were active on Twitter. The main items discussed in the chapter are: \emph{i}) the statistics related to the diffusion of hashtags representative of conversations on HPC and QT projects, and \emph{ii}) comparison of the aforementioned statistics to assess the Twitter audience interested in QTs.
\end{description}

\noindent
This work was done under the supervision of Dr. Daniela Ovadia. Questions and comments can be sent to giulio.mazzolo@gmail.com