\chapter*{Abstract}
This thesis presents an analysis of the online communication activity of EU-funded research projects in Future and Emerging Technologies (FET) within the Horizon 2020 programme. In particular, it focuses on the use of the Twitter platform by FET projects in high performing computing (HPC) and quantum technologies (QT).

FET projects were found to be present on disparate social media. More than 95\% of the projects have created a website. The most used social networks are: Twitter ($\sim 50\%$ of the projects), Facebook ($\sim 20\%$) and LinkedIn ($\sim 15\%$). Finally, the analysis suggests that the number of online channels considered by each FET project is not strongly influenced by the available budgets, but rather by the pursued communication strategy.

HPC projects are among the most socially active initiatives within the FET environment. In particular, roughly 80\% of them have created a profile on Twitter. Among HPC projects, the average posting rate is of the order of one tweet per week. The Twitter activity of HPC initiatives is typically comparable to that of the FET profile in terms of \textit{i}) number of account's tweets retweeted by other users, and \textit{ii}) average number of shared links per tweet (approximately 30\% and 0.4, respectively). Currently, the most influential HPC projects on Twitter have roughly more than 100 followers and a ratio between followers and followed accounts larger than 2. Finally, conversations mentioning HPC projects were found to be extremely viral and capable of reaching hundreds of thousands of users.

Contrarily to HPC projects, no QT initiative has created an account on Twitter. An analysis was performed to estimate the potential reach of QT projects on this social platform. The investigation was based on the amount of mentions of specific hashtags. The result indicates that QT projects may reach a Twitter community comparable to that of HPC initiatives. \\

\noindent
\textbf{Key words:} science communication, high performing computing, quantum technologies, Future and Emerging Technologies, Horizon 2020, social media.