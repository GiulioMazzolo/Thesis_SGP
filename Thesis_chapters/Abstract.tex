\chapter*{Abstract}
This thesis presents a search on the use of social media for communication and dissemination purposes made by EU-funded research projects in Future and Emerging Technologies (FET) within the Horizon 2020 programme. In particular, it focuses on the activity on the Twitter platform of FET projects in high performing computing (HPC) and quantum technologies (QT).

FET projects were found to be present on disparate social media. More than 95\% of the projects have created a website. The most used social networks are: Twitter (fraction of projects present on this platform ranging between ... and ...), Facebook (between ... and ...) and LinkedIn (between ... and ...). The value intervals are set by the lack of data for some projects. Finally, the number of online communication channels considered by each FET project was found not to be strongly influenced by the available budgets.

HPC projects are among the most socially active initiatives within the FET environment. In particular, roughly 80\% of them have created a profile on Twitter. Among HPC projects, the average posting rate is of the order of one tweet per week. The Twitter activity of HPC initiatives is typically comparable to that of the FET account in terms of number of profile tweets retweeted by other users and of average number of shared links per tweet (approximately 30\% and 0.4, respectively). Currently, the most influential HPC projects on Twitter have roughly more than 100 followers and a ratio between followers and followed accounts larger than roughly 2. Finally, conversations mentioning HPC projects were found to be extremely viral and capable of reaching hundreds of thousands of users.

Contrarily to HPC projects, no QT initiative has created an account on Twitter. An analysis was performed to estimate the potential reach of QT projects on this social platform. The investigation was based on the volume of mentions of specific hashtags. The result indicates that QT projects may reach an amount of users on Twitter comparable to that of HPC initiatives. \\

\noindent
\textbf{Key words:} high performing computing, quantum technologies, Future and Emerging Technologies, Horizon 2020, social media.