\chapter{The role of science communication}
This chapters presents a very conincise overview of the importance of science communication in modern society. 

\section{The knowledge era}
Three economical eras have been identified in the history of human civilisation \cite{Aasi1}. The first one was the agricultural age. This is believed to have started between 10000 and 8000 B.C. in different regions in the world. The second era is the industrial one. It began in England in the 18th century as a result of the industrial revolution. The third era is the knowledge age, and it is the one into which human civilization is currently entering. 

The three eras are based on different primary production resources. For the agricultural age, these were the work of people and animals. In the case of industry-driven economy, the ultimate source of richness and development is the work of people and machines. On the contrary, the knowledge era is not founded on the capacity to produce and accumulate tangible goods, but rather on the ability to store, generate and apply new information.

The information on which the knowledge age is based is mainly scientific. In the past centuries, the impact of science on humanity has been growing without interruption. Nowadays, the outcomes of scientific activities permeate our society and heavily shape our life style. Examples range from telecommunications to cancer cures, or from artificial intelligence to the development of new materials.

The reason for the increasing impact of science is the peculiar nature of knowledge as a resource. Just like any other resource, it is important for its capacity to provide solutions to problems. However, contrarily to resources such as water, gold or oil, scientific knowledge is potentially unlimited, as it is capable of generating itself (new knowledge leads to more knowledge). Moreover, the same knowledge can be used simultaneously by multiple entities. Hence, scientific knowledge is intrinsically a non-exclusive good.

For its characteristics as a resource, scientific knowledge has revolutionised the global economy. The ongoing, science-driven change of the world market has introduced countless positive innovations. However, it also has lead to dramatic societal and environmental changes.   

\section{Science and society}

In fact, the majority of the world population does not benefit from the results of scientific research. This is  to the tendency of private investors to the intrinsic non-exclusivity of scientific knowledge for the

peculiar nature of knowledge is due to Esempi sul fatto che costo dipende da conoscenza e investimenti di nazioni

Societa ricca ma con forti disuguaglianza

Lorem ipsum dolor sit amet, consectetur adipisci elit, sed eiusmod tempor incidunt ut labore et dolore magna aliqua. Ut enim ad minim veniam, quis nostrum exercitationem ullam corporis suscipit laboriosam, nisi ut aliquid ex ea commodi consequatur. Quis aute iure reprehenderit in voluptate velit esse cillum dolore eu fugiat nulla pariatur. Excepteur sint obcaecat cupiditat non proident, sunt in culpa qui officia deserunt mollit anim id est laborum.

Lorem ipsum dolor sit amet, consectetur adipisci elit, sed eiusmod tempor incidunt ut labore et dolore magna aliqua. Ut enim ad minim veniam, quis nostrum exercitationem ullam corporis suscipit laboriosam, nisi ut aliquid ex ea commodi consequatur. Quis aute iure reprehenderit in voluptate velit esse cillum dolore eu fugiat nulla pariatur. Excepteur sint obcaecat cupiditat non proident, sunt in culpa qui officia deserunt mollit anim id est laborum.

Lorem ipsum dolor sit amet, consectetur adipisci elit, sed eiusmod tempor incidunt ut labore et dolore magna aliqua. Ut enim ad minim veniam, quis nostrum exercitationem ullam corporis suscipit laboriosam, nisi ut aliquid ex ea commodi consequatur. Quis aute iure reprehenderit in voluptate velit esse cillum dolore eu fugiat nulla pariatur. Excepteur sint obcaecat cupiditat non proident, sunt in culpa qui officia deserunt mollit anim id est laborum.

Lorem ipsum dolor sit amet, consectetur adipisci elit, sed eiusmod tempor incidunt ut labore et dolore magna aliqua. Ut enim ad minim veniam, quis nostrum exercitationem ullam corporis suscipit laboriosam, nisi ut aliquid ex ea commodi consequatur. Quis aute iure reprehenderit in voluptate velit esse cillum dolore eu fugiat nulla pariatur. Excepteur sint obcaecat cupiditat non proident, sunt in culpa qui officia deserunt mollit anim id est laborum.

\section{Second section}
Lorem ipsum dolor sit amet, consectetur adipisci elit, sed eiusmod tempor incidunt ut labore et dolore magna aliqua. Ut enim ad minim veniam, quis nostrum exercitationem ullam corporis suscipit laboriosam, nisi ut aliquid ex ea commodi consequatur. Quis aute iure reprehenderit in voluptate velit esse cillum dolore eu fugiat nulla pariatur. Excepteur sint obcaecat cupiditat non proident, sunt in culpa qui officia deserunt mollit anim id est laborum.