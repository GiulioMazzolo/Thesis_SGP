\chapter{The role of science communication}
This chapters presents a very concise overview of the importance of science communication in modern society. 

\section{The knowledge era} \label{The_knowledge_era}
Three economical eras have been identified in the history of human civilisation \cite{Aasi1}. The first one was the agricultural age. This is believed to have started between 10000 and 8000 B.C. in different regions in the world. The second era is the industrial one. It began in England in the 18th century as a result of the industrial revolution. The third era is the knowledge age, and it is the one into which human civilization is currently entering. 

The three eras are based on different primary production resources. For the agricultural age, these were the work of people and animals. In the case of industry-driven economy, the ultimate source of richness and development is the work of people and machines. On the contrary, the knowledge era is not founded on the capacity to produce and accumulate tangible goods, but rather on the ability to store, generate and apply new information.

The information on which the knowledge age is based is mainly scientific. In the past centuries, the impact of science on humanity has been growing without interruption. Nowadays, the outcomes of scientific activities permeate our society and heavily shape our life style. Examples range from telecommunications to cancer cures, or from artificial intelligence to the development of new materials.

The reason for the increasing impact of science is the peculiar nature of knowledge as a resource. Just like any other resource, it is important for its capacity to provide solutions to problems. However, contrarily to resources such as water, gold or oil, scientific knowledge is potentially unlimited, as it is capable of generating itself (knowledge leads to new knowledge). Moreover, the same knowledge can be used simultaneously by multiple entities. Hence, scientific knowledge is intrinsically a non-exclusive good.

For its characteristics as a resource, scientific knowledge has revolutionised the economy on a planetary level. The ongoing, science-driven change of the global market has introduced countless positive innovations. However, it has also lead to dramatic societal changes.   

\section{Challenges in the knowledge era}
The relationship between scientific research and society has changed significantly after the second world war. From the half of the past century, several countries have started using science and its generation of new knowledge and technology as a source of economical growth. This process has progressively become more intense over the past decades. Nowadays, the national economies showing the highest growth rates are those of countries investing significant fraction of their gross domestic product in research and development. Examples are the United States, Northern Europe and Asiatic countries such as China, India and South Korea.   

The capacity of scientific knowledge to generate richness has attracted a growing number of private investors. As a result, in many countries private investments on scientific research are larger than public funds. One example are the United States, where private funding is twice as large as the public one. 

The leading role of private investors is based on a reinterpretation of knowledge as a resource. To pursue personal profit, investors are typically non interested in sharing the knowledge they have developed or they way they have used it to create goods. This approach limits the possibility to generate new knowledge from the results of others. Moreover, those with limited buying power cannot afford specific classes of products and hence cannot benefit from the knowledge behind them. One example are patented expensive medicines. In such a scenario, knowledge as a resource partially looses the intrinsic characteristics of being unlimited and non exclusive mentioned in section \ref{The_knowledge_era}. 

The current knowledge-driven development of the global economy has two important consequences. The first one is that humanity is richer than ever before. The second one is the growing societal inequality caused by the progressive concentration of the generated richness in the hands of few individuals. 

The increasing inequality is an obstacle for the creation of a democratic society. The scenario humanity is facing can be changed if knowledge will not be used as a mere instrument of power, but rather as a common good everyone should benefit from. This paradigm shift is one of main goals of the scientific citizenship.

\section{The scientific citizenship}
Lorem ipsum dolor sit amet, consectetur adipisci elit, sed eiusmod tempor incidunt ut labore et dolore magna aliqua. Ut enim ad minim veniam, quis nostrum exercitationem ullam corporis suscipit laboriosam, nisi ut aliquid ex ea commodi consequatur. Quis aute iure reprehenderit in voluptate velit esse cillum dolore eu fugiat nulla pariatur. Excepteur sint obcaecat cupiditat non proident, sunt in culpa qui officia deserunt mollit anim id est laborum.