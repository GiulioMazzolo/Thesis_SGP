\chapter*{Conclusions}
This thesis focused on the online aspects of the communication campaigns designed by FET research projects. The results indicate that FET initiatives do consider social media as an opportunity to reach stakeholders and the general public. Nevertheless, the use of online channels is not uniform across the disparate FET investigation lines.  

One example is offered by HPC and QT research projects. The two classes face similar communication challenges and pursue partially-overlapping objectives (although the strategies to achieve them are very different). Despite such similarities, the approaches followed by HPC and QT projects are opposite. On one hand, HPC initiatives make an effective use of popular social media such as Twitter and Facebook. On the other hand, QT projects do not base their communication efforts on online platforms. In particular, they totally disregard Twitter.  

The motivation for the different behaviour lies probably in the fact that the development of QTs is still in the initial phase. Hence, it is not known yet whether these technologies will indeed be achieved and applications to society are not to be expected in the near future. Nevertheless, the limited use of social media made by QT projects reduces significantly the reachable audience and slows down. In fact, the potential reach of QT initiatives was estimated to be comparable to the online audience of HPC projects, which consists of hundreds of thousands of users. Expanding the community coming across with the QT investigation line draws attention on the overall effort and may attract investors and societal uptake.

A more effective use of social media may be expected from the FET Flagship initiative on QTs. This programme will be launched in 2018 and is one of the major investigation efforts ever undertaken by the European Union. In particular, the   it can be believed that a better effor by current QT projects would have prepared a better environment for the flagship effort and offered guidelines