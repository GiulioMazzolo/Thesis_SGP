\chapter*{Conclusions}
This thesis focused on the online aspects of the communication campaigns designed by FET research projects. The results indicate that FET initiatives do consider social media as an opportunity to reach stakeholders and the general public. Nevertheless, the use of online channels is not uniform across disparate FET investigation lines.  

One example is offered by HPC and QT research projects. The two classes face similar communication challenges and pursue partially overlapping objectives (although the strategies to achieve them are very different). Despite such similarities, the approaches followed by HPC and QT projects are opposite. On one hand, HPC initiatives make an effective use of popular social media such as Twitter and Facebook. On the other hand, QT projects do not base their communication efforts on online platforms. In particular, they disregard Twitter.  

The motivation for the different behaviour lies probably in the fact that the development of QTs is still in the initial phase. It is not known yet whether this effort will indeed be successful. Hence, applications to society are not to be expected in the near future. 

Nevertheless, the limited use of social media made by QT projects reduces significantly the societal uptake of this effort. The potential reach of QT initiatives was estimated to be comparable to the online audience of HPC projects, which consists of hundreds of thousands of users. Expanding the community coming in contact with the QT investigation line would draw attention on this potentially groundbreaking scientific frontier and may attract funding.

A more active use of social media may be expected from the FET Flagship initiative on QTs. This is one of the major investigation efforts ever undertaken by the European Union and will be launched in 2018. However, a strong presence of QT projects on social platforms prior to the beginning of the flagship initiative may have \textit{i}) provided useful hints and guidelines on how to best design an effective QT communication strategy, and \textit{ii}) contributed to building a preexisting engaged community.