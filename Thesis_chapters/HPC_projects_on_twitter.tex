\chapter{HPC projects on Twitter} \label{HPC_projects_on_Twitter}
As shown in section \ref{Online_presence_breakdown}, roughly 80\% of the FET HPC projects have created a profile on Twitter. This percentage is larger than the corresponding value calculated for the other FET projects. Nevertheless, it is insufficient to determine whether HPC projects are conducting active communication campaigns on Twitter. 

To assess the activity and influence of HPC projects on Twitter, an analysis of their profiles was performed. The results were integrated by the monitoring activity of the mentions to HPC projects on Twitter over a period of three and a half months. The goal of this second analysis was to assess the virality of tweets on the considered projects.  

Both analyses are described in this chapter. Section \ref{Overall_activity} provides an overview of the past activity of the HPC Twitter profiles. Section \ref{Most_influential_projects} identifies the most influential HPC projects. Section \ref{Mentions_of_HPC_profiles} presents the monitoring of the mentions to HPC projects. Section ... ranks HPC project in order of virality of their mentions. 

\section{Overall activity} \label{Overall_activity}
\afterpage{
 \clearpage %Flush earlier floats (otherwise order might not be correct)
 \thispagestyle{empty} %empty page style
   \begin{landscape}
   %\begin{table}[htb]
   \begin{table}
   \begin{adjustwidth}{-1.5cm}{}
   {\tiny
    \begin{tabular}{*{8}{c}} 
      \hline 
  	  \hline
       Project & Date of first tweet & Tweets & Tweets per day & Tweets retweeted & Times per retweeted tweet & Links per tweet & Hashtags per tweet \\ 
       \hline
       \hline
       ALLScale & 26/05/2016 & 39 & 0.08 & 15\% & 1.67 & 0.72 & 0.38 \\
       ANTAREX & 25/09/2015 & 24 & 0.03 & 37\% & 1.56 & 0.63 & 0.04 \\
       COMPAT & 01/10/2015 & 122 & 0.16 & 7\% & 1.63 & 0.30 & 0.05 \\
       DEEP-EST & 19/05/2014 & 900 & 0.72 & 40\% & 2.08 & 0.52 & 1.59 \\
       ECOSCALE & 17/10/2015 & 19 & 0.03 & 21\% & 1.25 & 0.26 & 0.00 \\
       EuroLab-4-HPC & - & 0 & 0 & 0\% & 0 & 0 & 0 \\
       ExaFLOW & 27/10/2015 & 389 & 0.54 & 24\% & 1.63 & 0.62 & 0.97 \\
       ExaNeSt & 29/11/2015 & 1 059 & 1.54 & 12.5\% & 1.38 & 0.46 & 0.06 \\
       ExaNoDe & 20/06/2017 & 38 & 0.32 & 13.2\% & 2.60 & 0.21 & 0.03 \\
       EXDCI & 30/03/2016 & 864 & 1.53 & 16\% & 2.90 & 0.20 & 0.23 \\
       EXTRA & 06/10/2015 & 4 & 0.01 & 0\% & 0 & 0.25 & 0.25 \\
       INTERTWINE & 28/11/2016 & 99 & 0.31 & 51.5\% & 2.18 & 0.77 & 0.79 \\
       MANGO & 03/12/2015 & 32 & 0.05 & 43.8\% & 1.93 & 0.38 & 0.38 \\
       Mont-Blanc 3 & 06/02/2012 & 2 506 & 1.21 & 23.6\% & 2.68 & 0.32 & 0.50 \\
       NEXTGenIO & 30/09/2015 & 211 & 0.28 & 23.7\% & 3.02 & 0.14 & 0.52 \\
       READEX & 13/10/2015 & 29 & 0.04 & 69.0\% & 1.60 & 0.62 & 1.03 \\
       SAGE & 30/09/2015 & 92 & 0.12 & 32.6\% & 1.77 & 0.20 & 0.07 \\ 
       FET & 07/01/2016 & 3 199 & 4.94 & 32.3\% & 4.46 & 0.42 & 0.92 \\
       \hline
       \hline
    \end{tabular}
   }     
   \caption{Statistics collected from the Twitter accounts of the HPC projects. The data were collected from the date of the project's first tweet to 14th October 2017. Tweets per day is the average number of tweets posted each day. Tweets retweeted corresponds to the fraction of the project's tweets which have been retweeted by other accounts. Times per retweeted tweet refers to the average number of times a retweeted post has been retweeted. The last two columns report the average number of links and hashtags per project's tweet. The EuroLab-4-HPC project has posted no tweets since the creation of the account. The last row refers to the @fet\textunderscore eu profile of the FET funding programme. For this account, the statistics are limited to the maximum number of past tweets returned by Twitter (3200). The data were collected with the Twitter Analytics Tool Twitonomy.} \label{HPC_Twitter_activity}
   \end{adjustwidth} 
   \end{table}
   \end{landscape}
 \clearpage
}

The past activity of the Twitter accounts of HPC projects was investigated with the Twitter Analytics Tool Twitonomy \cite{Twitonomy}. Analysed data cover the profiles' histories since their creation and till 14th October 2017. The statistics collected for each project are listed in table \ref{HPC_Twitter_activity}. The results calculated from these data are outlined in the following subsections.

\begin{figure}
 \centering
 \begin{subfigure}[t]{0.9\textwidth}
   \includegraphics[width=1\linewidth]{Images/Tweets_Exanest.png}
   \caption{} 
 \end{subfigure}

 \begin{subfigure}[t]{0.9\textwidth}
   \includegraphics[width=1\linewidth]{Images/Tweets_Exdci.png}
   \caption{}
 \end{subfigure}
 \caption{(a) Time distribution of the number of tweets posted by ExaNeSt, the HPC project with the largest average number of tweets per day (1.54) as of 14th October 2017. (b) As for (a) but for EXDCI, the HPC project with the second largest average number of tweets per day (1.53). The plots were generated with the Twitter Analytics Tool Twitonomy.} 
 \label{Tweets_Exanest-Exdci}
\end{figure}

\begin{figure}
 \centering
 \begin{subfigure}[t]{0.9\textwidth}
   \includegraphics[width=1\linewidth]{Images/Tweets_Montblanc.png}
   \caption{} 
 \end{subfigure}

 \begin{subfigure}[t]{0.9\textwidth}
   \includegraphics[width=1\linewidth]{Images/Tweets_Deepest.png}
   \caption{}
 \end{subfigure}
 \caption{(a) Time distribution of the number of tweets posted by Mont-Blanc 3, the HPC project with the third largest average number of tweets per day (1.21) as of 14th October 2017. (b) As for (a) but for DEEP-EST, the HPC project with the fourth largest average number of tweets per day (900). The plots were generated with the Twitter Analytics Tool Twitonomy.} 
 \label{Tweets_Montblanc-Deepest}
\end{figure}

\subsubsection{Tweets per day}
Out of the seventeen considered HPC projects, three have an average tweeting rate larger than one post per day. The time distribution of the tweets of the accounts with the highest average rates are shown in figures \ref{Tweets_Exanest-Exdci} and \ref{Tweets_Montblanc-Deepest}. The tweeting rate is lower than once every second day for twelve profiles. In particular, one project has posted no tweets since the creation of the account. 

The median of the projects' rates is 0.16 tweets/day. This corresponds to roughly 5 posts per month. The median was chosen as representative value of the HPC posting rates for its robustness in the presence of outliers, see also section \ref{Budget_impact}.

\subsubsection{Retweets}
The percentage of an account's tweets retweeted by other users offers an estimate of the effectiveness of its activity on Twitter. The higher the percentage, the more the account is considered a valuable source of information by the Twitter community. 

The HPC project with the largest fraction of tweets retweeted by other accounts is READEX (roughly 70\%). Except for one, all other profiles have a percentage value smaller than 50\%. The median of the fraction of retweeted tweets calculated over all HPC projects is 23.6\%.

The results on the percentage of retweeted posts is integrated by the average number of times such tweets were retweeted by different users. The higher this value, the more the Twitter community finds the profile's tweets worth to be forwarded. Table \ref{HPC_Twitter_activity} shows that six projects have an average number of times of retweet higher or equal to two. The result indicates that posts are typically retweeted by more than one user. The median calculated over all profiles is 1.67.

\subsubsection{Links and hashtags}
Links and hashtags are effective ways to enhance the relevance of a Tweet. In particular, the higher the average number of links per tweet for a given profile, the more likely the account is a source of information to other users. The higher the average number of hashtags per tweet, the higher the chance that the profile's tweets are found in a search.

As shown in table \ref{HPC_Twitter_activity}, six HPC accounts have an average number of links per tweets higher or equal to 0.5. This corresponds to one link every second tweet. The median calculated over all HPC Twitter accounts is 0.32, i.e. one link every third tweet. Three projects have accounts with an average number of hashtags equal or larger than one. The median is 0.25, which corresponds to one hashtag every fourth tweet.

\subsubsection{Comparison to the FET Twitter profile}
Table \ref{HPC_Twitter_activity} reports the statistics calculated for the Twitter profile @fet\textunderscore eu of the FET funding programme as well. Data were collected starting from 7th January 2016. The date was determined by the maximum number of past tweets returned by Twitter (3200). 

The corresponding average number of tweets per day is roughly thirty times larger than the median of HPC projects. This is probably due to the largest resources available to the FET initiative compared to single HPC projects. Nevertheless, the fraction of retweeted tweets is not significantly larger than the median calculated for HPC (32\% vs 23.6\%). The same holds for the average number of links per tweets (0.42 vs 0.32), whereas the average number of hashtags is roughly four times larger (0.92 vs 0.25).

\section{Most influential projects} \label{Most_influential_projects}
There are disparate ways to estimate the influence of Twitter profiles. One is based on both \textit{i}) the number of followers, and \textit{ii}) the ratio between the number of followers and the number of accounts followed by the considered profile (following). Influential users are identified by a large community of followers and a high ratio followers/following.

\begin{table}[t]
 \begin{center}
 {\scriptsize
  \begin{tabular}{cccc}
   \hline 
   \hline
   Project & Followers & Following & Followers/Following \\ 
   \hline
   \hline
   ALLScale & 41 & 28 & 1.46 \\
   ANTAREX & 77 & 13 & 5.92 \\
   COMPAT & 131 & 160 & 0.82 \\
   DEEP-EST & 697 & 534 & 1.31 \\
   ECOSCALE & 42 & 1 & 42 \\
   EuroLab-4-HPC & 24 & 2 & 12 \\
   ExaFLOW & 206 & 90 & 2.29 \\
   ExaNeSt & 211 & 261 & 0.81  \\
   ExaNoDe & 52 & 54 & 0.96 \\
   EXDCI & 405 & 169 & 2.40 \\
   EXTRA & 45 & 18 & 2.50\\
   INTERTWINE & 106 & 59 & 1.80 \\
   MANGO & 74 & 46 & 1.61 \\
   Mont-Blanc 3 & 1 420 & 687 & 2.07 \\
   NEXTGenIO & 162 & 44 & 3.86 \\
   READEX & 116 & 55 & 2.11 \\
   SAGE & 122 & 86 & 1.42 \\ 
   FET & 6 499 & 1 612 & 4.03 \\
   \hline
   \hline
  \end{tabular}
 } 
 \end{center} 
 \caption{Number of followers and followed accounts (following) on Twitter for the HPC projects as of 14th October 2017. Influential profiles are identified by high numbers of followers and high values of the ratio between followers and following. Data were collected with the Twitter Analytics Tool Twitonomy.}
\label{HPC_influence_table} 
\end{table}

Table \ref{HPC_influence_table} lists the number of followers and of following for each HPC project, together with the ratio followers/following. Data were collected on 4th October 2017 with the Twitonomy application. The medians of the number of followers and of the values followers/following are equal to 116 and 2.07, respectively. 

\begin{figure}[!t] 
 \begin{center}
 \includegraphics[scale=0.4]{Images/HPC_influence.png}
 \caption{Distribution of HPC projects as a function of the number of followers and of the ratio between followers and followed accounts (following) on Twitter. The data used for the plot were collected on 14th October 2017 with the Twitter Analytics Tool Twitonomy. The dashed lines identify the medians of the number of followers and of the values of the ratio followers/following calculated over the HPC profiles. The most influential projects are located in the upper right quarter (high number of followers and high values of followers/following). For the sake of clarity, the figure shows the follower follower/following ranges up to 750 and 6, respectively. The following projects were used to calculated the medians but lie outside the plotted ranges: ECOSCALE (42 followers, follower/following equal to 42), EuroLab-4-HPC (24 followers, follower/following equal to 12) and Mont-Blanc 3 (1420 followers, follower/following equal to 2.07).}
 \label{HPC_influence_plot}
 \end{center}
\end{figure}

To identify the most influential profiles among HPC projects, the following analysis was performed. First, projects were distributed on a plane as a function of the number of followers and of the ratio followers/following, see figure \ref{HPC_influence_plot}. Second, the plane was divided into four regions by drawing the lines corresponding to the aforementioned medians of the number of followers and of the values followers/following. The most influential HPC projects fall in the quarter of the plane identified by the conditions that the number of followers is larger than 116 and the ratio followers/following is larger than 2.07.

By following this procedure, the most influential projects are ExaFLOW (206, 2.29), EXDCI (405, 2.40), Mont-Blanc 3 (1420, 2.07) and NEXTGenIO (162, 3.86). The READEX profile (116, 2.11) is representative of the influence of HPC projects, as its values lie very close to the calculated medians.

\section{Mentions of HPC profiles} \label{Mentions_of_HPC_profiles} 
Another approach to estimate the influence of HPC projects on Twitter consists of monitoring the mentions of their accounts (i.e., @AllScaleEurope, @antarex\textunderscore project, @compatproject etc.). The monitoring activity covered the time period from 1st July to 12th October 2017 and was performed with the Twitter Analytics Tool NUVI \cite{NUVI}. Monitored mentions sum up to 1323. 

\begin{figure}[!t] 
 \begin{center}
 \includegraphics[scale=0.4]{Images/NUVI_time_distribution.png}
 \caption{Time distribution of the mentions of HPC profiles on Twitter between 1st July and 12th October 2017. The plot was created with the Twitter Analytics Tool NUVI.}
 \label{NUVI_time_distribution}
 \end{center}
\end{figure}

The time distribution of the mentions is shown in figure \ref{NUVI_time_distribution}. The peak of conversation (59 mentions) happened on 12th September 2017. During the peak, the most frequently used keywords were filippo mantovani, workshop, server cpu, prototype and processors. An overview of the topics treated in the monitored mentions is provided in figure \ref{HPC_word_burst}. The figure is based on the 1005 mentions which came across 485 major categories over the monitored time. The list of most shared words is in table \ref{Most_shared_words}.

\begin{table}[t]
 \begin{center}
 %{\scriptsize
  \begin{tabular}{cccc}
   \hline 
   \hline
   Shared word & Mentions & Fraction of total mentions \\ 
   \hline
   \hline
   amp & 22 & 2.2\% \\
   project & 18 & 1.8\% \\
   supercomputer & 16 & 1.6\% \\
   application & 15 & 1.5\% \\
   etp4h & 14 & 1.4\% \\
   compute & 11 & 1.1\% \\
   \hline
   \hline
  \end{tabular}
 %} 
 \end{center} 
 \caption{List of the most shared words in the 1005 mentions of the HPC profiles which came across the 485 major categories considered by the Twitter Analytics Tool NUVI. The second and third columns list the amount of mentions in which the considered word was shared and the percentage of the considered mentions. The 1005 mentions are a subset of the 1323 monitored with NUVI between 1st July and 12th October 2017.}
\label{Most_shared_words} 
\end{table}

\begin{figure}[!t] 
 \begin{center}
 \includegraphics[scale=0.55]{Images/HPC_word_burst.png}
 \caption{Projects' distribution as a function of the available budget and of the number of considered online communication channels. The vertical lines are the budget medians of the group of projects with activated channels ranging between one and four. For the sake of clarity, the figure shows the budget range up to \euro 11.5 Million. The following projects were used for the medians calculation but lie outside the plotted budget range: QuantERA (\euro 40.5 Million, 3 channels), FLAG-ERA II (\euro 18.3 Million, two channels) and DEEP-EST (\euro 15.9 Million, 3 channels).}
 \label{HPC_word_burst}
 \end{center}
\end{figure}

Out of the 1323 mentions, 637 were original mentions, which had the potential of reaching an audience of 172 720 users. This reach is calculated as the sum of the followers of the accounts mentioning the analysed keywords. Moreover, 116 unique profiles made a total of 686 reshares. Shares and re-tweets spread the mentions to an additional 233 974 people. The spread is calculated as the sum of the followers of the accounts which shared or retweeted the tweets with the mentions. Reach and spread together give an estimate of the potential audience which came across with the tweeted contents. Figure \ref{HPC_Most_reach_spread_popular} shows the mentions with the largest reach, spread and the most popular one. The ratio between reach and spread defines the viral coefficient, which is equal to 1.4, see figure \ref{HPC_viral_coefficient}. As the value of the viral coefficient is larger than one indicates that the mentions were extremely viral. 

\begin{figure}[!t] 
 \begin{center}
 \includegraphics[scale=0.41]{Images/HPC_Most_reach_spread_popular.png}
 \caption{Projects' distribution as a function of the available budget and of the number of considered online communication channels. The vertical lines are the budget medians of the group of projects with activated channels ranging between one and four. For the sake of clarity, the figure shows the budget range up to \euro 11.5 Million. The following projects were used for the medians calculation but lie outside the plotted budget range: QuantERA (\euro 40.5 Million, 3 channels), FLAG-ERA II (\euro 18.3 Million, two channels) and DEEP-EST (\euro 15.9 Million, 3 channels).}
 \label{HPC_Most_reach_spread_popular}
 \end{center}
\end{figure}

\begin{figure}[t] 
 \begin{center}
 \includegraphics[scale=0.2]{Images/HPC_viral_coefficient.png}
 \caption{Projects' distribution as a function of the available budget and of the number of considered online communication channels. The vertical lines are the budget medians of the group of projects with activated channels ranging between one and four. For the sake of clarity, the figure shows the budget range up to \euro 11.5 Million. The following projects were used for the medians calculation but lie outside the plotted budget range: QuantERA (\euro 40.5 Million, 3 channels), FLAG-ERA II (\euro 18.3 Million, two channels) and DEEP-EST (\euro 15.9 Million, 3 channels).}
 \label{HPC_viral_coefficient}
 \end{center}
\end{figure}

%\begin{figure}[!t] 
% \begin{center}
% \includegraphics[scale=0.55]{Images/HPC_word_burst.png}
% \caption{Projects' distribution as a function of the available budget and of the number of considered online communication channels. The vertical lines are the budget medians of the group of projects with activated channels ranging between one and four. For the sake of clarity, the figure shows the budget range up to \euro 11.5 Million. The following projects were used for the medians calculation but lie outside the plotted budget range: QuantERA (\euro 40.5 Million, 3 channels), FLAG-ERA II (\euro 18.3 Million, two channels) and DEEP-EST (\euro 15.9 Million, 3 channels).}
% \label{HPC_word_burst}
% \end{center}
%\end{figure}

\section{Most viral HPC projects}

\begin{table}[t]
 \begin{center}
 {\scriptsize
  \begin{tabular}{cccc}
   \hline 
   \hline
   Project & Reach & Spread & Viral coefficient \\ 
   \hline
   \hline
   ALLScale & 48 & 0 & 0.0 \\
   ANTAREX & 202 & 9 & 0.0 \\
   COMPAT & 13 286 & 112 & 0.0 \\
   DEEP-EST & 11 759 & 6 988 & 0.6 \\
   ECOSCALE & 55 & 2 268 & 41.2 \\
   EuroLab-4-HPC & 2 135 & 14 679 & 6.9 \\
   ExaFLOW & 18 366 & 28 311 & 1.5 \\
   ExaNeSt & 18 109 & 62 259 & 3.4  \\
   ExaNoDe & 6 559 & 4 470 & 0.7 \\
   EXDCI & 111 534 & 23 082 & 0.2 \\
   EXTRA & 216 & 24 & 0.1 \\
   INTERTWINE & 7 436 & 6 673 & 0.9 \\
   MANGO & 1 360 & 162 & 0.1 \\
   Mont-Blanc 3 & 30 362 & 58 782 & 1.9 \\
   NEXTGenIO & 4 056 & 53 838 & 13.5 \\
   READEX & 117 & 0 & 0.0 \\
   SAGE & 1 537 & 4 836 & 3.1 \\ 
   \hline
   \hline
  \end{tabular}
 } 
 \end{center} 
 \caption{Summary of the Twitter analytics for the hashtag \#quantumcomputing over the monitored time periods. The potential reach is defined as the total aggregate number of followers of the people who mentioned the considered keyword in their tweets.}
\label{HPC_viral_coefficients} 
\end{table}